\subsubsection{Pregunta 2: Determinar si algún compilador de C al que se tiene acceso implementa la función free.}

El compilador Cygwin asociado con Netbeans permite usar la función free que nos sirve para liberar espacios de memoria que hemos separado de forma dinámica con la función malloc, a continuación se presenta un ejemplo de su uso\\

\lstset{language = C} 
\begin{lstlisting}[frame = single] %Comienzo del Código
#include <stdio.h>
#include <stdlib.h>
#include <string.h>

 typedef enum {ROJO, BLANCO, NEGRO, AZUL} colores;
 
 typedef struct carro{
   colores color;
   int max_vel;
 } carro;
 
 int main(){
   int i,size;
   carro* myArray;
   carro* carro_actual; 
   
   printf("Ingrese el numero de carros: ");
   scanf("%d",&size);
   
   myArray= (carro*) malloc( size* sizeof(carro));

   for (i=0;i<size;i++)
   {
      carro_actual=(myArray + i );
      carro_actual->color= i%4;
      carro_actual->max_vel= i;
   }
   
   free(myArray);
   return 0;
}

\end{lstlisting}






