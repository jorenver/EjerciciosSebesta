\subsubsection{Pregunta 9:Escriba un programa en C + +, Java o C # que muestra el orden de evaluación de las expresiones utilizadas como parámetros actuales a un método.}


Codigo:\\

\lstset{language = Java} 
\begin{lstlisting}[frame = single] %Comienzo del Código

class A{
    int a=20;
    int b=20;
    
    int multiplicar(int x){
        x=50;
        return 2*x;
    }
    
    void calculo(){
        a=a+multiplicar(a);
        b=multiplicar(b)+b;
        System.out.println("valor de a:"+a);
        System.out.println("valor de b:"+b);
    }
    
}


public class JavaApplication {
    
    public static void main(String[] args) {
        A a=new A();
        a.calculo();
    }
}

\end{lstlisting}

Resultados:
El resultado de las dos operaciones efectuadas dentro de la funcion calculo es exactamente el mismo, en este caso, la evaluacion se realiza de izquierda a derecha, ya que los variables son pasadas por valor, lo que ocurre dentro de la funcion no afecta a las variables a y b, no existen cambios
Al final se imprime en pantalla: valor de a: 120 valor de b:120

