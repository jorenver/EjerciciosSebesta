\subsubsection{Pregunta 5:  Escriba una función en C que incluya la siguiente  secuencia de declaraciones:\\
x = 21;\\
int x;\\
x = 42;\\
Ejecute  el programa y explique los resultados. Reescriba el mismo códifgo en  C++
y Java y compare los resultados.}

Escrito en C:\\

\lstset{language = C} 
\begin{lstlisting}[frame = single] %Comienzo del Código
void main(){
	int x = 21;
	funcion_Prueba(x);
	printf("%i",x);
	getch();
}


void funcion_Prueba(int numero){
	int x;
	x = 42;
	numero=x;
}
\end{lstlisting}
RESULTADOS: 21\\\
Como se puede apreciar, el valor de la variable x se mantiene en 21, ser pasada como argumento en la función funciónPrueba no alteró su valor a pesar de que ocurre una igualación a 42 en numero=x; esto se debe a que en C la variable ésta siendo pasada por valor, por lo cual se crea una copia ésta.\\ 
Ahora veamos como reaccionán Java y C++:

Escrito en Java:\\
\lstset{language = Java}
\begin{lstlisting}[frame = single] %Comienzo del Código
public class Prueba{
	public static void main(String[] args){
		int x=21;
		funcion_Prueba(x);
		System.out.printf("%d",x);
	}
	public static void funcion_Prueba(int numero){
		int x;
		x = 42;
		numero=x;
	}
}
\end{lstlisting}

RESULTADOS: 21\\\


Escrito en C++:\\
\lstset{language = C++}
\begin{lstlisting}[frame = single] %Comienzo del Código
int main(){
	int x=21;
	funcion_Prueba(x);
	std::cout<<x;
	getch();
	return 0;
}

void funcion_Prueba(int numero){
	int x;
	x = 42;
numero=x;
}
RESULTADOS: 21\\\
Ambos resultados son iguales, esto era de esperarse ya que tanto Java y C++, se basan en C.
\end{lstlisting}

