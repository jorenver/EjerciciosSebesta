\documentclass[12pt,oneside]{article}
\usepackage{geometry}                                % See geometry.pdf to learn the layout options. There are lots.
\geometry{a4paper}                                           % ... or a4paper or a5paper or ... 
%\geometry{landscape}                                % Activate for for rotated page geometry
%\usepackage[parfill]{parskip}                    % Activate to begin paragraphs with an empty line rather than an indent
\usepackage{graphicx}                                % Use pdf, png, jpg, or epsß with pdflatex; use eps in DVI mode
                                                                % TeX will automatically convert eps --> pdf in pdflatex                
\usepackage{amssymb}

\usepackage[spanish]{babel}                        % Permite que partes automáticas del documento aparezcan en castellano.
\usepackage{listings}
\usepackage[utf8]{inputenc}                        % Permite escribir tildes y otros caracteres directamente en el .tex
\usepackage[T1]{fontenc}                                % Asegura que el documento resultante use caracteres de una fuente apropiada.

\usepackage{hyperref}                                % Permite poner urls y links dentro del documento

\title{Ejercicios de Programación - Sebesta}
\author{Lenguajes de Programación - ESPOL}
%\date{}                                                        % Activate to display a given date or no date

\begin{document}
\maketitle

\section{Introducción}
Las respuestas propuestas en este repositorio son producto del trabajo de los estudiantes de la materia ``Lenguajes de Programación'' de la ESPOL, correspondientes a las preguntas del libro de Robert Sebesta, Concepts of Programming Languages.

\section{Preguntas y Respuestas}

\subsection{Capítulo 5: Nombres, Enlaces y Alcances}
\subsubsection{Pregunta 4: Escriba un script en Python con subprogramas triplemente anidados, donde cada subprograma referencie a variables que han sido definidas en otros niveles de la jerarquía}

Escriba su respuesta con claridad. En las secciones de código utilice listings.

\lstset{language = Python} 
\begin{lstlisting}[frame = single] %Comienzo del Código
def operar ():
	x=3
	y=5
	def operar2():
		z=x-2
		a=5*y
		def operar3():
			i=x+y-5
			j=x+2*y	
	z=z*y
	a=z+i
	j=j-x
\end{lstlisting}

\subsubsection{Pregunta 5:  Escriba una función en C que incluya la siguiente  secuencia de declaraciones:\\
x = 21;\\
int x;\\
x = 42;\\
Ejecute  el programa y explique los resultados. Reescriba el mismo códifgo en  C++
y Java y compare los resultados.}

Escrito en C:\\

\lstset{language = C} 
\begin{lstlisting}[frame = single] %Comienzo del Código
void main(){
	int x = 21;
	funcion_Prueba(x);
	printf("%i",x);
	getch();
}


void funcion_Prueba(int numero){
	int x;
	x = 42;
	numero=x;
}
\end{lstlisting}
RESULTADOS: 21\\\
Como se puede apreciar, el valor de la variable x se mantiene en 21, ser pasada como argumento en la función funciónPrueba no alteró su valor a pesar de que ocurre una igualación a 42 en numero=x; esto se debe a que en C la variable ésta siendo pasada por valor, por lo cual se crea una copia ésta.\\ 
Ahora veamos como reaccionán Java y C++:

Escrito en Java:\\
\lstset{language = Java}
\begin{lstlisting}[frame = single] %Comienzo del Código
public class Prueba{
	public static void main(String[] args){
		int x=21;
		funcion_Prueba(x);
		System.out.printf("%d",x);
	}
	public static void funcion_Prueba(int numero){
		int x;
		x = 42;
		numero=x;
	}
}
\end{lstlisting}

RESULTADOS: 21\\\


Escrito en C++:\\
\lstset{language = C++}
\begin{lstlisting}[frame = single] %Comienzo del Código
int main(){
	int x=21;
	funcion_Prueba(x);
	std::cout<<x;
	getch();
	return 0;
}

void funcion_Prueba(int numero){
	int x;
	x = 42;
numero=x;
}
RESULTADOS: 21\\\
Ambos resultados son iguales, esto era de esperarse ya que tanto Java y C++, se basan en C.
\end{lstlisting}


\subsubsection{Pregunta 6: Escriba programas de prueba en C + +, Java, C \# y para determinar el alcance de una variable declarada en una sentencia for. Específicamente, el código debe determinar si una variable es visible después del cuerpo de la declaración for.}

Java:  Este ejemplo llena un arreglo de 10 elementos con los números del 1 al 10, usando la variable uno declarada en el cuerpo del for, luego se pretende imprimir dicha variable, en un lugar fuera del for.\\

\lstset{language = Java} 
\begin{lstlisting}[frame = single] %Comienzo del Código
public class Prueba{
	public static void main(String[] args){
		int arreglo[];
		arreglo=new int[10];
		for(int i=0;i<10;i++){
			int uno=1;
			arreglo[i]=uno+i;
		}
		System.out.printf("%i",uno);
	}
}
}
\end{lstlisting}
RESULTADOS: \\
Prueba.java:12: error: cannot find symbo\\l
 symbol:   variable uno\\
 location: class Prueba\\
1 error\\

Error de compilación, la variable no es visible esta fuera del alcance.\\
\\
En C++\\
Usaré el mismo ejemplo anterior, para probar:\\


\lstset{language = C++} 
\begin{lstlisting}[frame = single] %Comienzo del Código
int main(){
	int i;
	int arreglo[10];
	
	for(i=0;i<10;i++){
		int uno=1;
		arreglo[i]=uno+i;
	}
	std::cout<<uno;
	getch();
	return 0;
}
\end{lstlisting}


RESULTADOS:\\ 
error C2065: 'uno' : identificador no declarado\\
De la misma manera, ocurre un error de compilación porque la variable no es visible.

Ejemplo en C \# : \\

\lstset{language = C++} 
\begin{lstlisting}[frame = single] %Comienzo del Código
namespace PruebaCchar
{
    class Program
    {
        static void Main(string[] args)
        {
            int i;
            int[] arreglo = new int[10];
	
	        for(i=0;i<10;i++){
		        int uno=1;
		        arreglo[i]=uno+i;
	        }
            Console.Write(uno);
            Console.ReadKey();
        }
    }
}
\end{lstlisting}

RESULTADOS:\\ 
Error de compilación: El nombre 'uno' no existe en el contexto actual\\

En los tres lenguajes sucede que la variable que es declarada dentro de un for no es visible en otro sitio.


\subsubsection{Write three functions in C or C++: one that declares a large array stati-cally, one that declares the same large array on the stack, and one that  creates the same large array from the heap. Call each of the subprograms a large number of times (at least 100,000) and output the time required by each. Explain the results}

%Escriba su respuesta con claridad. En las secciones de código utilice listings.

\lstset{language = C} 
\begin{lstlisting}[frame = single] %Comienzo del Código
include <stdio.h>
include <time.h>
define MAX 1000
void Stack();
void Heap();
void main(){
	int i ,t1,t2,t3,dt1,dt2;
	t1 = time(NULL);
	printf("%d\n",t1);
	for (i = 0; i < 100000; i++)
		Stack();
	t2 = time(NULL);
	printf("%d\n", t2);
	for (i = 0; i < 100000; i++)
		Heap();
	t3 = time(NULL);
	printf("%d\n", t3);
	dt1 = t2 - t1;
	dt2 = t3 - t2;
	printf("%d    %d\n", dt1, dt2);
	while (getchar() != '\n')
		getchar();
}

void Stack(){
	int i;
	int Arreglo[MAX];
	for (i = 0; i < MAX; i++)
		Arreglo[i] = i;
}

void Heap(){
	int *Arr;
	int i;
	Arr = (int*)malloc(sizeof(int)*MAX);
	for (i = 0; i < MAX; i++)
		Arr[i] = i;
}

\end{lstlisting}
despues de ejecutar el codigo  nos damos cuenta que el tiempo en operar sobre el arreglo declarado en el Heap es mucho mayor al declarado en el Stack. Esto se debe ha que las variables almacenadas en el Stack estan mucho mas organizadas que las del Heap y accesarlas es mucho mas rapido

	

\subsection{Capítulo 7: Expresiones y asignaciones}
\subsubsection{Pregunta 2: Reescriba el programa de Programming Exercise 1 en  C++, Java, y C \#, ejecútelos, y  compare los resultados.}


Ejemplo escrito en C++:\\

\lstset{language = C++} 
\begin{lstlisting}[frame = single] %Comienzo del Código
int main(){
	int i = 10, j = 10, sum1, sum2;
	sum1 = (i / 2) + fun(&i);
	sum2 = fun(&j) + (j / 2);
	std::cout<<sum1;
	std::cout<<sum2;
	getch();
	return 0;
}

int fun(int *k) {
*k += 4;
return 3 * (*k) - 1;
}

}
\end{lstlisting}
RESULTADOS: \\
sum1 = 46
sum2 = 48\\
Los operandos son evaluados de izquierda a derecha, en el caso de sum1 el resultados es 46 porque primero de evalúa 'i/2' igual a 5, y luego se evalúa la función la cuál  retorna 41 porque el valor de la variable i sigue siendo 10. Por otro lado sum2 toma el valor de 48 porque primero se evalúa 'fun(\&j)'  esto retorna 41, pero el valor de 'j' cambia a 14, y al efectuar 'j/2' retorna 7, y al sumar da 48.\\

En java:\\


\lstset{language = C++} 
\begin{lstlisting}[frame = single] %Comienzo del Código
public class Prueba{
	public static void main(String[] args){
		int sum1,sum2;
		Integer i,j;
		i=new Integer(10);
		j=new Integer(10);
		sum1=(i.intValue()/2) + fun(i);
		sum2=fun(j)+ (j.intValue()/2);
		System.out.printf("%d\n%d",sum1,sum2);
		}
	public static int fun(Integer k){
		k=k.intValue()+4;
		return 3*(k.intValue())-1;
		
	}
}
\end{lstlisting}

RESULTADOS: \\
sum1 = 46
sum2 = 46\\

Con este código no se puede concluir si los operandos son evaluados de izquierda a derecha o viceversa porque sum1 y sum2 toman el mismo valor 46. Pero se puede concluir que una referencia a un objeto es pasado por valor en una función, porque el valor de las variables 'i' y 'j' no se vió afectada.\\

En C \#:


\lstset{language = C++} 
\begin{lstlisting}[frame = single] %Comienzo del Código


namespace PruebaCchar
{
    class Program
    {
        unsafe static void Main(string[] args)
        {
            int i = 10, j = 10, sum1, sum2;
            sum1 = (i / 2) + fun(&i);
            sum2 = fun(&j) + (j / 2);
            Console.Write(sum1);
            Console.Write(sum2);
            Console.ReadKey();
        }

       unsafe public static int fun(int *k){
            *k += 4;
            return 3 * (*k) - 1;
        }
    }
}

\end{lstlisting}

RESULTADOS: \\
sum1 = 46
sum2 = 48\\
Los operandos se evalúan de izquierda a derecha, tiene el mismo comportamiento que C++

\subsubsection{Write a test program in your favorite language that determines and outputs the precedence and associativity of its arithmetic and Boolean operators.}

%Escriba su respuesta con claridad. En las secciones de código utilice listings.

\lstset{language = C} 
\begin{lstlisting}[frame = single] %Comienzo del Código
void precedencia{
	int a,b,c,d,r;
	a=3;
	b=5;
	c=9
	d=15;
	r=a+c*d;
	printf("precedencia Aritmetica\n")
	printf("\na=3\nb=5\nc=9\nd=15");
	printf("a+b-c*d= %d\n",r);
	printf("a*b+c-d= %d\n",r);
	printf("a*b-c/d= %d\n",r);
	printf("a/b+c*d= %d\n",r);
	
	printf("Precedencia Boleana\n");
	printf(" 1 && 0 || 1 %d\n",(1 && 0 || 1));
	printf(" 1 || 0 && 1 %d\n",(1 || 0 && 1));
	printf(" ! 0 && 1 %d\n",(! 0 && 1));
	
	
}
	
\end{lstlisting}

\subsubsection{Write a Java program that exposes Java’s rule for operand evaluation 
order when one of the operands is a method call.}

Escriba su respuesta con claridad. En las secciones de código utilice listings.

\lstset{language = Java} 
\begin{lstlisting}[frame = single] %Comienzo del Código
public class Prueba {
	float a;
	float b;
	float c;
	float d; 
	public Prueba(float a , float b , float c, float d){
		this.a=a;
		this.b=b;
		this.c=c;
		this.d=d;
	}
	float metodo(){
		a=8;
		b=10;
		return 2;
	}
	
	public static void main(String args[]){
		Prueba p = new Prueba(15,5,3,9);
		
		Systen.out.println("\n a=15 \n b=5\n c=3 \n d=9");
		Systen.out.println("\n a+b*c-d= "+(p.a+p.b*p.c-p.d));
		Systen.out.println("\n a+b*c-d= "+(p.a+p.b*p.c-p.d));
		Systen.out.println("\n a+b/c-d= "+(p.a+p.b/p.c-p.d));
		Systen.out.println("\n a+b*c+metodo()= "+(p.a+p.b*p.c+p.metodo()));
		Systen.out.println("\n metodo()+a+b*c= "+(p.metodo()+p.a+p.b*p.c));
	}
}
	
\end{lstlisting}


\subsection{Capítulo 8: intrucciones y nivel de control de estructuras}
\subsubsection{Pregunta 4: Considere el siguiente segmento de un programa C. Reescríbalo sin usar gotos o breaks.}

\lstset{language = C++} 
\begin{lstlisting}[frame = single] %Comienzo del Código
s.
j = -3;
for (i = 0; i < 3; i++) {
switch (j + 2) {
case 3:
case 2: j--; break;
case 0: j += 2; break;
default: j = 0;
}
if (j > 0) break;
j = 3 - i;
}
\end{lstlisting}

Reescrito:\\

\lstset{language = C++} 
\begin{lstlisting}[frame = single] %Comienzo del Código
void main(){
int i,j = -3;
	for (i = 0; i < 3; i++) {
	switch (j + 2) {
	case 3:
	case 2: j--;
	case 0: j += 2;
	default: j = 0;
	}
	if (j > 0)
	j = 3 - i;
	}


	printf("%i\n%i",j,i);
}
\end{lstlisting}

Al inicio los valores de 'i' y 'j' fueron 1 y 3 respectivamente, al borrar todos los break los valores cambiaron a 0 y 3 respectivamente, Porque en C, al no incluir un break al terminar un case, continuara con el otro case. Es un muye sencillo darse cuenta de porque sucede esto, basta con hacer una prueba de escritorio.\\


\subsubsection{Pregunta 5: En una carta al editor de CACM, Rubin (1987) se utiliza el siguiente segmento de código como evidencia de que la legibilidad de algún código con gotos es mejor que el código equivalente sin gotos. Este código busca la  primera fila de una matriz de enteros de n por n llamada X que  tiene nada más que valores cero; Reescribe este codigo sin gotos en uno de los siguientes lenguajes: C, C++, Java, C \#, o Ada. Compare la legibilidad de su código con la del código del ejemplo.
}

\lstset{language = C++} 
\begin{lstlisting}[frame = single] %Comienzo del Código
for (i = 1; i <= n; i++) {
for (j = 1; j <= n; j++)
if (x[i][j] != 0)
goto reject;
println ('First all-zero row is:', i);
break;
reject:
}

\end{lstlisting}

Reescrito e Java:\\

\lstset{language = C++} 
\begin{lstlisting}[frame = single] %Comienzo del Código
for (int i = 1; i<=n; i++){
for (int j = 1; j <=n; j++){
if( x [ i ] [ j ] !=0 ) {
System.out.println(  i );
break;
}
}
}
\end{lstlisting}

Comparando la legibilidad, la segunda opción me parece más legible porque en la primera se dificulta un poco la lectura del flujo del programa, en la segunda opción se puede leer más fácilmente.
% Continuar con los siguientes capítulos y ejercicios:
% Ch6: 1, 2, 7
% Ch7: 1 - 6, 9
% Ch8: 3, 4, 5
% Ch9: 1, 5
% Recuerden que todos corresponden a las secciones de "Programming Exercises".

\end{document}
